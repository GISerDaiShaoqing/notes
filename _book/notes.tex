\PassOptionsToPackage{unicode=true}{hyperref} % options for packages loaded elsewhere
\PassOptionsToPackage{hyphens}{url}
%
\documentclass[]{book}
\usepackage{lmodern}
\usepackage{amssymb,amsmath}
\usepackage{ifxetex,ifluatex}
\usepackage{fixltx2e} % provides \textsubscript
\ifnum 0\ifxetex 1\fi\ifluatex 1\fi=0 % if pdftex
  \usepackage[T1]{fontenc}
  \usepackage[utf8]{inputenc}
  \usepackage{textcomp} % provides euro and other symbols
\else % if luatex or xelatex
  \usepackage{unicode-math}
  \defaultfontfeatures{Ligatures=TeX,Scale=MatchLowercase}
\fi
% use upquote if available, for straight quotes in verbatim environments
\IfFileExists{upquote.sty}{\usepackage{upquote}}{}
% use microtype if available
\IfFileExists{microtype.sty}{%
\usepackage[]{microtype}
\UseMicrotypeSet[protrusion]{basicmath} % disable protrusion for tt fonts
}{}
\IfFileExists{parskip.sty}{%
\usepackage{parskip}
}{% else
\setlength{\parindent}{0pt}
\setlength{\parskip}{6pt plus 2pt minus 1pt}
}
\usepackage{hyperref}
\hypersetup{
            pdftitle={Notes},
            pdfauthor={Shaoqing Dai},
            pdfborder={0 0 0},
            breaklinks=true}
\urlstyle{same}  % don't use monospace font for urls
\usepackage{longtable,booktabs}
% Fix footnotes in tables (requires footnote package)
\IfFileExists{footnote.sty}{\usepackage{footnote}\makesavenoteenv{longtable}}{}
\usepackage{graphicx,grffile}
\makeatletter
\def\maxwidth{\ifdim\Gin@nat@width>\linewidth\linewidth\else\Gin@nat@width\fi}
\def\maxheight{\ifdim\Gin@nat@height>\textheight\textheight\else\Gin@nat@height\fi}
\makeatother
% Scale images if necessary, so that they will not overflow the page
% margins by default, and it is still possible to overwrite the defaults
% using explicit options in \includegraphics[width, height, ...]{}
\setkeys{Gin}{width=\maxwidth,height=\maxheight,keepaspectratio}
\setlength{\emergencystretch}{3em}  % prevent overfull lines
\providecommand{\tightlist}{%
  \setlength{\itemsep}{0pt}\setlength{\parskip}{0pt}}
\setcounter{secnumdepth}{5}
% Redefines (sub)paragraphs to behave more like sections
\ifx\paragraph\undefined\else
\let\oldparagraph\paragraph
\renewcommand{\paragraph}[1]{\oldparagraph{#1}\mbox{}}
\fi
\ifx\subparagraph\undefined\else
\let\oldsubparagraph\subparagraph
\renewcommand{\subparagraph}[1]{\oldsubparagraph{#1}\mbox{}}
\fi

% set default figure placement to htbp
\makeatletter
\def\fps@figure{htbp}
\makeatother

\usepackage{booktabs}
\usepackage[]{natbib}
\bibliographystyle{apalike}

\title{Notes}
\author{Shaoqing Dai}
\date{2020-06-08}

\begin{document}
\maketitle

{
\setcounter{tocdepth}{1}
\tableofcontents
}
\hypertarget{ux5b66ux4e60ux7b14ux8bb0ux6982ux8ff0}{%
\chapter*{学习笔记概述}\label{ux5b66ux4e60ux7b14ux8bb0ux6982ux8ff0}}
\addcontentsline{toc}{chapter}{学习笔记概述}

这一份笔记是我在看到\href{https://yufree.cn/}{于淼}的\href{http://yufree.github.io/notes/}{笔记}以后,感悟颇多,因此也开始学习建立这个项目。这份笔记我会记录三个内容,一个是我所上公开课的笔记,一个是我学习相关书籍的笔记,最后一个是我听部分讲座的笔记。本人感兴趣的内容颇多,加上本身地理学综合性学科背景出身,因此涉及到的主题会相对分散和广泛,希望可以给大家带来一些帮助。

在此也感谢\href{https://yihui.org/}{谢益辉}开发的\href{https://bookdown.org/}{bookdown}包\citep{xie2015},本站笔记基于bookdown构建。

\hypertarget{ux8054ux7cfbux65b9ux5f0f}{%
\section*{联系方式}\label{ux8054ux7cfbux65b9ux5f0f}}
\addcontentsline{toc}{section}{联系方式}

\begin{longtable}[]{@{}ll@{}}
\toprule
Author & Shaoqing Dai\tabularnewline
\midrule
\endhead
E-mail & \href{mailto:dsq1993qingge@163.com}{\nolinkurl{dsq1993qingge@163.com}}\tabularnewline
Homepage & \url{https://gisersqdai.top/mycv/}\tabularnewline
Blog & \url{http://gisersqdai.top/}\tabularnewline
\bottomrule
\end{longtable}

\hypertarget{ux7b14ux8bb0ux76eeux5f55}{%
\section*{笔记目录}\label{ux7b14ux8bb0ux76eeux5f55}}
\addcontentsline{toc}{section}{笔记目录}

\begin{itemize}
\tightlist
\item
  01 Esir MOOC Cartography (制图学)
\end{itemize}

\hypertarget{cartography}{%
\chapter{Esir MOOC Cartography (制图学)}\label{cartography}}

\bibliography{book.bib,packages.bib}

\end{document}
